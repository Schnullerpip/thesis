\chapter{Grundlagen: Scala/Xtext}
\section{Scala}
Im folgenden soll zum Verständnis zunächst die Programmiersprache Scala vorgestellt werden.
Die Entwicklung von Scala (Scalable Language) begann bereits 2001 an der École polytechnique fédérale de Lausanne in der Schweiz von einem Team um Professor Martin Odersky. Designziel war eine elegante, typsichere Programmierung, vollkommene Kompatibilität mit der JVM und eine Zusammenführung der Objektorientierung und der funktionalen Programmierung (vgl. \cite{braun:scala}). Scala ist hierbei objektorientierter als z.B. Java, da in Scala \textit{alles} ein Objekt ist. Odersky selber bezeichnet Scala als postfunktionale Sprache, da immernoch diskutiert wird, ob Scala als funktionale Programmiersprache betitelt werden darf. Fest steht jedoch, dass man in Scala funktional programmieren kann - man muss es allerdings nicht. Funktionen können Argumente oder Ergebnisse anderer Funktionen sein und sind somit \textit{Higher Order Functions}. Trotz der Typsicherheit bietet Scala das \textit{Feeling} einer dynamisch typisierten Sprache, da Typen beim Überetzen inferiert werden können. Der Compiler erkennt also anhand des initialisierten Wertes, um was für einen Typ es sich handelt, so kann fast immer auf manuelle Typkennzeichnung verzichtet werden.
\begin{lstlisting}[style = scala]
val name = "Julian"
val alter = 24
\end{lstlisting}
sind vollkommen korrekte Ausdrücke und werden ebenso kompiliert wie:
\begin{lstlisting}[style = scala]
val name:String = "Frederik"
val alter:Int = 33
\end{lstlisting}
Scala ist einfacher zu erlernen, als beispielsweise Java, da Anweisungen auch alleinstehend als Skript oder direkt in einem interaktiven Scalainterpreter ausgeführt und ausprobiert werden können. 
Komplexe Problemstellungen können stark abstrahiert in wenigen Zeilen Code beschrieben werden. Über entsprechende Methoden, kann man sogar eigene Operatoren und Kontrollstrukturen erstellen \textit{"`Damit ist Scala prädestiniert zur Erstellung von Domain Specific Languages $[$...$]$"'.} \citet[p. 2]{braun:scala}
%TODO Beispiel
\section{Xtext}\label{xtext}
Nachdem nun Scala erörtert wurde, soll im folgenden die zu ersetzende Technologie \textit{Xtext} erklärt werden.
\textit{"`Xtext ist ein Framework zum Erstellen von Programmiersprachen und \textbf{D}omain-\textbf{S}pecific \textbf{L}anguages. Mit Xtext wird deine Programmiersprache durch eine mächtige Gramatiksprache definiert"'} $\sim$ \citet{xtext:website}.
Xtext ist ein Framework mit dem u.a. Domänenspezifische Sprachen entwickelt werden können und ergänzt das \textit{Eclipse-Modeling-Framework}\cite{emf}. Dafür wird eine komplexe Grammatiksprache zur Vergügung gestellt, über die Entitäten deklariert und assoziiert werden können. Anhand der Grammatik erzeugt Xtext Parser, Generatoren und sogar ein Klassenmodell für die beschriebenen Klassen (vgl. \citet{xtext:goodbyexml}). Um die beschriebene \textit{DSL} ausführbar zu machen wird außerdem ein Texteditor generiert, welcher entsprechendes Syntax Highlighting automatisch programmiert. Der große Vorteil von Xtext ist, dass die Definition der beschriebenen Sprache gleichzeitig die Objektstruktur der zu generierenden Modelle beschreibt. Wird die Objektstruktur der Modelle verändert, ändert sich die DSL ebenfalls und so bleibt die Abbildung von DSL auf Modell immer konsistent. 
%TODO: Hinführung,in welchen Bereichen wird Xtex eingesetzt, Beispiel, Positives/Negatives
